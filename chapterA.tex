\chapter{希腊字母常用指代意义及其中文读音}

\begin{longtable}[]{@{}m{0.8cm}m{0.8cm}m{0.8cm}m{2.2cm}m{1.5cm}m{1.6cm}m{6cm}@{}}
\toprule


\textbf{序号} & \textbf{大写} & \textbf{小写} & \textbf{英语音标注音} &
\textbf{英文} & \textbf{汉语名称} & \textbf{常用指代意义}\tabularnewline
\midrule
\endhead
1 & $A$ & $\alpha$ & /'ælfə/ & alpha & 阿尔法 &
角度、系数、角加速度、电离度、转化率\tabularnewline
2 & $B$ & $\beta$ & /'bi:tə/ 或 /'beɪtə/ & beta & 贝塔 &
磁通系数、角度、系数\tabularnewline
3 & $\Gamma$ & $\gamma$ & /'gæmə/ & gamma & 伽玛 &
电导系数、角度、比热容比\tabularnewline
4 & $\Delta$ & $\delta$ & /'deltə/ & delta & 得尔塔 &
\textbf{变化量}、焓变、熵变、屈光度、\textbf{一元二次方程中的判别式}、化学位移\tabularnewline
5 & E & $\epsilon$ & /'epsɪlɒn/ & epsilon & 艾普西隆 &
对数之基数、\textbf{介电常数}、\textbf{电容率}、应变\tabularnewline
6 & Z & $\zeta$ & /'zi:tə/ & zeta & 泽塔 &
系数、方位角、阻抗、相对黏度\tabularnewline
7 & H & $\eta$ & /'i:tə/ & eta & 伊塔 & 迟滞系数、机械效率\tabularnewline
8 & $\Theta$ & $\theta$ & /'$\theta$i:tə/ & theta & 西塔 & 温度、角度\tabularnewline
9 & I & $\iota$ & /aɪ'əʊtə/ & iota & 约(yāo)塔 & 微小、一点\tabularnewline
10 & K & $\kappa$ & /'kæpə/ & kappa & 卡帕 & 介质常数、绝热指数\tabularnewline
11 & $\Lambda$ & $\lambda$ & /'læmdə/ & lambda & 拉姆达 & \textbf{波长}、体积、导热系数
普朗克常数\tabularnewline
12 & M & $\mu$ & /mju:/ & mu & 谬 &
磁导率、微、\textbf{动摩擦系(因)数}、流体动力黏度、货币单位,莫比乌斯函数\tabularnewline
13 & N & $\nu$ & /nju:/ & nu & 纽 &
磁阻系数、流体运动粘度、光波频率、化学计量数\tabularnewline

14 & $\Xi$ & $\xi$ &希腊 /ksi/
英美 /ˈzaɪ/ 或 /ˈsaɪ/&xi&克西&随机变量、(小)区间内的一个未知特定值\tabularnewline

15&O&o&/əuˈmaikrən/或 /ˈɑmɪˌkrɑn/&omicron&奥米克戎& 高阶无穷小函数\tabularnewline
16 &$\Pi$&$\pi$& /paɪ/ & pi & 派 &
圆周率、π(n)表示不大于n的质数个数、连乘\tabularnewline
17 &P&$\rho$& /rəʊ/ & rho & 柔 &
\textbf{电阻率}、柱坐标和极坐标中的极径、\textbf{密度}、曲率半径\tabularnewline
18 & $\Sigma$ & $\sigma$ & /'sɪɡmə/ & sigma & 西格马 &
总和、表面密度、跨导、应力、电导率\tabularnewline
19 & T & $\tau$ & /tɔ:/ 或 /taʊ/ & tau & 陶 &
时间常数、切应力、2$\pi$(两倍圆周率)\tabularnewline

20 & $\Upsilon$ & υ$\upsilon$ & /ˈipsɪlon/
 或 /ˈʌpsɪlɒn/  & upsilon  &  阿普西龙 &  位移 \tabularnewline
21 & $\Phi$ & $\varphi$ & /faɪ/ & phi & 斐 &
\textbf{磁通量}、电通量、角、透镜焦度、热流量、\textbf{电势}、直径、欧拉函数\tabularnewline
22 & X & $\chi$ & /kaɪ/ & chi & 希 &
统计学中有卡方($\chi$\^{}2)分布\tabularnewline
23 & $\Psi$ & $\psi$ & /ps/ & psi & 普西 &
角速、介质电通量、ψ函数、磁链\tabularnewline
24 & $\Omega$ & $\omega$ & /'əʊmɪɡə/
或 /oʊ'meɡə/ & omega & \begin{minipage}[t]{0.12\columnwidth}\raggedright
奥米伽

欧米伽\strut
\end{minipage} &
欧姆、角速度、角频率、交流电的电角度、化学中的质量分数、不饱和度\tabularnewline
\bottomrule
\end{longtable}
