\chapter{高中物理公式}
%%%
\begin{center}
	\textbf{必修一\&必修二}
\end{center}
\begin{longtable}[]{@{}|m{1.07cm}|m{2.43cm}|m{4cm}|m{5.5cm}|@{}}
	\hline
	\textbf{模块} & \textbf{概念/规律} & \textbf{公式 } & \textbf{备注}\endhead
	\hline
	\multirow{3}{1cm}{匀\\
	变\\
	速\\
	直\\
	线\\
	运\\
	动}
	&
	匀变速
	
	直线运动&
	$v_t=v_0+at$

	$\Delta x=aT^2$

	$\bar{v}=v_{\frac{t}{2}}=\dfrac{v_0+v_t}{2}=\dfrac{x}{t}$

	$x=v_0t+\dfrac{1}{2}at^2$

	$2ax=v_t^2-v_0^2$
	&
	$\bar{v}=\dfrac{x}{t}$适用于任何形式的运动\tabularnewline
	\cline{2-4}
	&自由落体
	
	运动&
	$v=gt$

	$h=\dfrac{1}{2}gt^2$

	$v^2=2gh$&
	$v_0=0$

	$a=g$
	\tabularnewline
	\cline{2-4}
	&竖直抛体
	
	运动&
	$v_t=v_0\pm gt$

	$h=v_0t\pm\dfrac{1}{2}gt^2$

	$v_t^2=v_0^2\pm 2ax$
	&
	上抛取-号

	下抛取+号\tabularnewline
	\cline{1-4}
	\multirow{3}{1cm}{相\\
	互\\
	作\\
	用}
	&
	重力&$G=mg$&\tabularnewline
	\cline{2-4}
	&胡克定律&$F=kx$&x 为形变量

	k 为劲度系数\tabularnewline
	\cline{2-4}
	&滑动摩擦力&$f=\mu F_N$&$\mu$为动摩擦因数\tabularnewline
	\cline{1-4}
	牛顿运动定律&牛顿第二定律&$F=ma$&F 为合外力,a 与 F 方向一致\tabularnewline
	\cline{1-4}
	\multirow{2}{1cm}{曲\\
	线\\
	运\\
	动}
	&平抛运动&
	$v_x=v_0$\quad 
	$v_y=gt$

	$x=v_0t$\quad
	$y=\dfrac{1}{2}gt^2$
	&沿 x 方向做匀速直线运动;
	
	沿 y 方向做自由落体运动\tabularnewline
	\cline{2-4}
	&匀速圆周运动&
	$v=\frac{\Delta s}{\Delta t}$

	$\omega=\frac{\Delta \theta}{\Delta t}$

	$a=\dfrac{v^2}{r}=\omega^2r=\omega v$

	$F=ma=m\dfrac{v^2}{r}=m\omega^2r$
	&
	$v=\dfrac{2\pi r}{T}$

	$\omega=\dfrac{2\pi}{T}$

	$v=\omega r$\tabularnewline
	\cline{1-4}
	万有引力定律&万有引力定律&$F_\text{万}=G\dfrac{Mm}{r^2}$&万有引力常量

	$G= 6.67 \times 10^{−11}N·𝑚^2/kg^2$\tabularnewline
	\cline{1-4}
	\multirow{3}{1cm}{功}
	&功 &$W=Fl\cos\alpha$&$\alpha$是 F 与 l 的夹角\tabularnewline
	\cline{2-4}
	&功率&
	平均功率$P=\dfrac{W}{t}$

	瞬时功率$P=Fv\cos\alpha$
	&$\alpha$是 F 与 v 的夹角\tabularnewline
	\cline{2-4}
	&机械功率&$\eta =\frac{W_\text{有用}}{W_\text{总}}\times 100\% $
	
	$\eta =\frac{P_\text{有用}}{P_\text{总}}\times 100\%$
	&$\eta<1$\tabularnewline
	\cline{1-4}
	\multirow{4}{1cm}{能}
	&
	动能&$E_k=\dfrac{1}{2}mv^2$&标量,具有对称性\tabularnewline
	\cline{2-4}
	&重力势能&$E_p=mgh$&与零势能面的选择有关\tabularnewline
	\cline{2-4}
	&动能定理&$W=\dfrac{1}{2}mv_2^2-\dfrac{1}{2}mv_1^2$&W为合外力做的功\tabularnewline
	\cline{2-4}
	&机械能守恒定律&
	$E_1=E_2$

	$E_{k1}+E_{p1}=E_{k2}+E_{p2}$
	&守恒条件:

	只有重力或弹力做功的物体系统内\tabularnewline
	\cline{1-4}
\end{longtable}
%%%
\begin{center}
	\textbf{选修3-1}
\end{center}
\begin{longtable}[]{@{}|m{1.07cm}|m{2.43cm}|m{4cm}|m{5.5cm}|@{}}
	\hline
	\textbf{模块} & \textbf{概念/规律} & \textbf{公式 } & \textbf{备注}\endhead
	\hline
	\multirow{7}{1cm}{静\\
	电\\
	场}
	&
	库仑定律&
	$F=k \frac{q_{1} q_{2}}{r^{2}}$

	$\left(k=9.0 \times 10^{9} N \cdot m^{2} / C^{2}\right)$
    &
	适用条件:静止在真空中的点电荷\tabularnewline
	\cline{2-4}
	&电场强度&
	$E=\frac{F}{q}$

	$E=k \frac{a}{r^{2}}$

	$E=\frac{U}{d}$
	&
    定义式:适用于任何电场的计算
    
    决定式:适用于真空中点电荷产生的电场

    关系式:适用于匀强电场
	\tabularnewline
	\cline{2-4}
	&电场力&
	$F=qE$
	&
	F与E的方向相同或相反\tabularnewline
    \cline{2-4}		
    &电势&
	$\varphi=\frac{E_{p}}{q}$
	&
	电势与0势能面的选取有关\tabularnewline
    \cline{2-4}		
    &电势差&
	$U_{A B}=\varphi_{A}-\varphi_{B}=\frac{W_{A B}}{q}$
	&
	电势差与0势能面的选取无关\tabularnewline
    \cline{2-4}	
    &电容器的电容&
	$c=\frac{Q}{U}$

	$C=\frac{\varepsilon_{r} S}{4 \pi k d}$
	&
	定义式
	
	决定式:适用于平行板电容器\tabularnewline
	\cline{1-4}
	\multirow{10}{1cm}{恒\\
	定\\
	电\\
	流}
	&
	电阻定律&$R=\rho \frac{l}{S}$&决定式:$\rho$为电阻率\tabularnewline
	\cline{2-4}
	&电流&$I=\frac{Q}{t}=n e S v$&x 为形变量

	k 为劲度系数\tabularnewline
	\cline{2-4}
	&电源电动势&$E=\frac{W}{q}=U_{\text {内 }}+U_{\text {外 }}$&\tabularnewline
    \cline{2-4}
		&欧姆定律&$I=\frac{U}{R}$
	
	$I=\frac{E}{R+r}$&部分电路欧姆定律

	闭合电路欧姆定律\tabularnewline
	\cline{2-4}
	&路端电压&U=E
	
	U=0
	
	$U=I R=E-U_{\text{内}}=E-I r$&断路时

	短路时
	
	通路时\tabularnewline
	\cline{2-4}
	&电功&$W=UIt$
	
	$W=I^{2} R t=\frac{U^{2}}{R} t$&适用于一切电路

	适用于纯电阻电路\tabularnewline
	\cline{2-4}
	&焦耳定律&$Q=I^{2} R t$&\tabularnewline
	\cline{2-4}
	&电源功率&$P=E I$&\tabularnewline
	\cline{2-4}
	&用电器功率&$P=\frac{W}{t}=U I$
	
	$P=I^{2} R=\frac{U^{2}}{R}$&适用于一切电路

	适用于纯电阻电路\tabularnewline
	\cline{2-4}
	&电源效率&$\eta=\frac{P_{\text{出}}}{P_{\text {总 }}} \times 100 \%$
	
	$=\frac{U_{\text {外 }}}{E} \times 100 \%$&\tabularnewline
	\cline{1-4}
	\multirow{3}{1cm}{磁\\
	场}
	&
	磁感应强度&
	$B=\frac{F}{IL}$
    &
	定义式:I与B的方向垂直\tabularnewline
	\cline{2-4}
	&安培力&
	$F=BIL$
	&
    $B \perp I$,方向:左手定则判断
	\tabularnewline
	\cline{2-4}
	&洛伦兹力&
	$F=qvB$
	&
	$B \perp v$,方向:左手定则判断\tabularnewline
    \cline{1-4}
\end{longtable}
%%%
\newpage
\begin{center}
	\textbf{选修3-2}
\end{center}
\begin{longtable}[]{@{}|m{1.07cm}|m{2.43cm}|m{4cm}|m{5.5cm}|@{}}
	\hline
	\textbf{模块} & \textbf{概念/规律} & \textbf{公式 } & \textbf{备注}\endhead
	\hline
	\multirow{3}{1cm}{电\\
	磁\\
	感\\
	应}
	&
	磁通量&
	$\Phi=B S \cos \theta$
    &
	$\theta$为平面S的垂线与B的夹角\tabularnewline
	\cline{2-4}
	&感应电动势&
	$E=n \frac{\Delta \phi}{\Delta t}$

	$E=B L v$
	&
    感生电动势
    
    动生电动势
	\tabularnewline
	\cline{1-4}
	\multirow{10}{1cm}{交\\
	变\\
	电\\
	流}
	&
	电动势(瞬时值)&$e=E_{m} \sin \omega t$&从中性面开始计时\tabularnewline
	\cline{2-4}
	&电流(瞬时值)&$i=I_{m} \sin \omega t$&从中性面开始计时\tabularnewline
	\cline{2-4}
	&电压(瞬时值)&$u=U_{m} \sin \omega t$&从中性面开始计时\tabularnewline
    \cline{2-4}
		&电动势(有效值)
		
		电流(有效值)
		
		电压(有效值)&$E=\frac{\sqrt{2}}{2} E_{m}$
	
		$I=\frac{\sqrt{2}}{2} l_{m}$
		
		$U=\frac{\sqrt{2}}{2} U_{m}$&适用于正弦式交变电流\tabularnewline
	\cline{2-4}
	&理想变压器&$\frac{n_{1}}{n_{2}}=\frac{U_{1}}{U_{2}}=\frac{I_{2}}{I_{1}}$
	
	$P_{1}=P_{2}$&
	\tabularnewline
	\cline{1-4}
\end{longtable}